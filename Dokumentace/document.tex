\documentclass[a4paper, 12pt]{report}
\usepackage{monapack}
\usepackage{siunitx}

%Proměnné
\student{Milan Jiříček}
\trida{B4.I}
\obor{18-20-M/01 Informační technologie}
\bydliste{Čenkov u Bechyně 3, 391 65 Bechyně}
\datumNarozeni{10. 11. 2001}
\vedouci{Ing. Břetislav Bakala}
\nazevPrace{Dálkové ovládání zásuvek NETIO}
\cisloPrace{12}
\skolniRok{2020/2021}
\reditel{Ing. Jiří Uhlík}

%Zadání
\zacatek

	\titulniStrana
	\anotace
	Maturitní práce se zaměřuje na porovnání platforem ESP8266 a ESP32. Cílem je vytvořit ovladač pro ovládání zásuvek značky NETIO s webovou aplikací pro konfiguraci a zjistit, která platforma je vhodna pro realizaci funkčního vzorku z hlediska spotřeby energie a~reakční doby.
	\annotation
	The graduation thesis focuses on the comparison of the ESP8266 and ESP32 platforms. The goal is to create a driver for controlling NETIO sockets with a web application for configuration and to find out which platform is suitable for the implementation of a~functional sample in terms of energy consumption and response time.
	\podekovani
	Chtěl bych poděkovat panu učiteli Ing.~Břetislavovi Bakalovi za odborné vedení práce a~cenné rady, které mi pomohly tuto práci zkompletovat. Rád bych také poděkoval Ing.~Břetislavovi Bakalovi za cenné rady, věcné připomínky a~vstřícnost při konzultacích a~vypracování bakalářské práce. V~neposlední řadě chci poděkovat Mgr. Haně Maříkové a~Mgr.~Vladimíře Špirhanzlové za pomoc při gramatické a~stylistické kontrole.
	\obsah
% Teorie
	\kapitola{Teorie}
		\podkapitola{Aplikace pro WiFi Managment}
		\podkapitola{Netio zásuvka Cobra}
		\podkapitola{a tak dale}
% MĚŘENÍ

	\kapitola{Měření spotřeby a času ESP8266}

		\podkapitola{Enable režim}

			\podpodkapitola{Klidový stav}
				\tucne{Podmínky}
				\bodseznam{
					\bod Napájení z USB
					\bod Měřeno pomocí úbytku napětí na rezistoru o velikosti \SI{10}{\ohm}
					\bod pin enable byl připojen manuálně
					\bod Napětí bylo měřeno Analog Discovery 2
				}
				\tucne{Výsledek} \\
				Po připojení ESP8266 proud nevzrostl a drží se stále na \SI{240}{\micro A}, což neodpovídá teoretickým hodnotám, které by se měly pohybovat okolo \SI{3}{\micro A}.
				\obrazek{1_enable}{Měření klidového režimu enable případu}{8cm}{1_enable.png}

			\podpodkapitola{WiFi připojení}
				\tucne{Podmínky}
				\bodseznam{
					\bod Měřeno pomocí úbytku napětí na rezistoru o velikosti \SI{0,7}{\ohm}
					\bod WiFi je nastavena pevně zadaná v programu
					\bod WiFi nevyužívá žádného zabezpečení
					\bod IP adresa byla nastavena staticky
				}

			\podpodkapitola{HTTP request}
				\tucne{Podmínky}
				\bodseznam{
					\bod Měřeno pomocí úbytku napětí na rezistoru o velikosti \SI{0,7}{\ohm}
					\bod WiFi je nastavena zachována v ESP z předchozího měření
				}
				\tucne{Výsledek}\\

			\podpodkapitola{Ohodnocení výsledků}
			Výsledky klidového režimu neodpovídají teoretické hodnotě uvedené v officiálním datasheetu. Důvodem je nízká citlivost zařízení Analog Discovery 2. Pro přesnější měření je žádoucí použít micro ampérmetr. \\
			Počáteční spuštění ESP8266 trvá déle než v ostatních případech. Hlavní důvod spočívá v rozdílném načítání než v případě deep sleep... Doplním


		\podkapitola{Deep sleep režim}
			\podpodkapitola{Klidový stav}
				\tucne{Podmínky}
					\bodseznam{
						\bod Napájení z USB
						\bod Měřeno pomocí úbytku napětí na rezistoru o velikosti \SI{10}{\ohm}
						\bod ESP8266 je probuzeno každých \SI{5}{s}
					}
				\tucne{Závěr měření}\\

			\podpodkapitola{WiFi připojení}
				\tucne{Podmínky}\\
				\tucne{Výsledek}\\

			\podpodkapitola{Odesílání HTTP requestu}
				\tucne{Podmínky}\\
				\tucne{Výsledek}\\

			\podpodkapitola{Ohodnocení výsledků}

		\podkapitola{Kontinuální režim}

			\podpodkapitola{Klidový stav}
				\tucne{Podmínky} \\
				\tucne{Výsledek} \\

			\podpodkapitola{WiFi připojení}
				\tucne{Podmínky}\\
				\tucne{Výsledek}\\

			\podpodkapitola{Odesílání HTTP requestu}
				\tucne{Podmínky}\\
				\tucne{Výsledek}\\

			\podpodkapitola{Ohodnocení výsledků}



% KONEC MĚŘENÍ
	\kapitola{Závěr}

	\seznamTabulek

	\seznamObrazku

	\prilohy{
		\kapitola{Příloha}
	}

	\literatura{

		\kniha{nazevCitace}{Příjmení autora}{Jméno autora}{Název knihy}{Místo vydání}{Nakladatelství}{Rok}{ISBN}

		\kvalifikacniprace{nazevCitace}{Příjmení autora}{Jméno autora}{Název práce}{Místo}{Rok}{Druh práce}{Univerzita, Fakulta, Katedra}{Vedoucí diplomové práce jméno}
	}

\konec
