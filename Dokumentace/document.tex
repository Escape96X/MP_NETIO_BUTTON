\documentclass[a4paper, 12pt]{report}
\usepackage{monapack}
\usepackage{siunitx}

%Proměnné
\student{Milan Jiříček}
\trida{B4.I}
\obor{18-20-M/01 Informační technologie}
\bydliste{Čenkov u Bechyně 3, 391 65 Bechyně}
\datumNarozeni{10. 11. 2001}
\vedouci{Ing. Břetislav Bakala}
\nazevPrace{Dálkové ovládání zásuvek NETIO}
\cisloPrace{12}
\skolniRok{2020/2021}
\reditel{Ing. Jiří Uhlík}

%Zadání
\zacatek
	
	\titulniStrana

	
	\anotace 
	Netio tlacitko a mereni

	\annotation
	netio button and measurement
	
	\podekovani
	husty podekovani
	

	
	\obsah
% Teorie


% MĚŘENÍ

	\kapitola{Měření spotřeby a času ESP8266}
		
		\podkapitola{Enable režim}
		
			\podpodkapitola{Klidový stav}
				\tucne{Podmínky}
				\bodseznam{
					\bod Napájení z USB
					\bod Měřeno pomocí úbytku napětí na rezistoru o velikosti \SI{10}{\ohm}
					\bod pin enable byl připojen manuálně
					\bod Napětí bylo měřeno Analog Discovery 2
				}
				\tucne{Výsledek} \\
				Po připojení ESP8266 proud nevzrostl a drží se stále na \SI{240}{\micro A}, což neodpovídá teoretickým hodnotám.
				\obrazek{1_enable}{Měření klidového režimu enable případu}{8cm}{1_enable.png}
				
			\podpodkapitola{WiFi připojení}
				\tucne{Podmínky}
				\bodseznam{
					\bod Měřeno pomocí úbytku napětí na rezistoru o velikosti \SI{0,7}{\ohm}
					\bod WiFi je nastavena pevně zadaná v programu
					\bod WiFi nevyužívá žádného zabezpečení
					\bod IP adresa byla nastavena staticky
				}
			
			\podpodkapitola{HTTP request}
				\tucne{Podmínky}
				\bodseznam{
					\bod Měřeno pomocí úbytku napětí na rezistoru o velikosti \SI{0,7}{\ohm}
					\bod WiFi je nastavena zachována v ESP z předchozího měření
				}
				\tucne{Výsledek}\\
				
			\podpodkapitola{Ohodnocení výsledků}
			Výsledky klidového režimu neodpovídají teoretické hodnotě uvedené v officiálním datasheetu. Důvodem je nízká citlivost zařízení Analog Discovery 2. Pro přesnější měření je žádoucí použít micro ampérmetr. \\
			Počáteční spuštění ESP8266 trvá déle než v ostatních případech. Hlavní důvod spočívá v rozdílném načítání než v případě deep sleep... Doplním
			
			
		\podkapitola{Deep sleep režim}
			\podpodkapitola{Klidový stav} 
				\tucne{Podmínky}
					\bodseznam{
						\bod Napájení z USB
						\bod Měřeno pomocí úbytku napětí na rezistoru o velikosti \SI{10}{\ohm}
						\bod ESP8266 je probuzeno každých \SI{5}{s}
					}
				\tucne{Závěr měření}\\
				
			\podpodkapitola{WiFi připojení}
				\tucne{Podmínky}\\
				\tucne{Výsledek}\\
				
			\podpodkapitola{Odesílání HTTP requestu}
				\tucne{Podmínky}\\
				\tucne{Výsledek}\\
				
			\podpodkapitola{Ohodnocení výsledků}
			
		\podkapitola{Kontinuální režim}
		
			\podpodkapitola{Klidový stav}
				\tucne{Podmínky}\\
				\tucne{Výsledek}\\
				
			\podpodkapitola{WiFi připojení}
				\tucne{Podmínky}\\
				\tucne{Výsledek}\\
				
			\podpodkapitola{Odesílání HTTP requestu}
				\tucne{Podmínky}\\
				\tucne{Výsledek}\\
				
			\podpodkapitola{Ohodnocení výsledků}
			


% KONEC MĚŘENÍ
	\kapitola{Závěr}

	\seznamTabulek
	
	\seznamObrazku
	
	\prilohy{
		\kapitola{Příloha}
	}
	
	\literatura{
		
		\kniha{nazevCitace}{Příjmení autora}{Jméno autora}{Název knihy}{Místo vydání}{Nakladatelství}{Rok}{ISBN}
		
		\kvalifikacniprace{nazevCitace}{Příjmení autora}{Jméno autora}{Název práce}{Místo}{Rok}{Druh práce}{Univerzita, Fakulta, Katedra}{Vedoucí diplomové práce jméno}
		
		\url{ESP Zapojení}{Read the Docs}{Boards}{readthedocs.org}{2017}{29.12.2020}{https://arduino-esp8266.readthedocs.io/en/latest/boards.html#generic-esp8266-module}
		
	}
	
\konec