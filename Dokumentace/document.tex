\documentclass[a4paper, 12pt]{report}
\usepackage{monapack}
\usepackage{siunitx}
\usepackage{caption}
\usepackage{graphicx}
\usepackage{subcaption}
\usepackage{hyperref}
% \usepackage{minted}
\usepackage[bottom]{footmisc}
\captionsetup[table]{skip=5pt}

%Proměnné
\student{Milan Jiříček}
\trida{B4.I}
\obor{18-20-M/01 Informační technologie}
\bydliste{Čenkov u Bechyně 3, 391 65 Bechyně}
\datumNarozeni{10. 11. 2001}
\vedouci{Ing. Břetislav Bakala}
\nazevPrace{Dálkové ovládání zásuvek NETIO}
\cisloPrace{12}
\skolniRok{2020/2021}
\reditel{Ing. Jiří Uhlík}

%Zadání
\begin{document}


	\titulniStrana
	\section*{Anotace}
		Maturitní práce se zaměřuje na porovnání platforem ESP8266 a ESP32. Cílem je vytvořit ovladač pro ovládání zásuvek značky NETIO s webovou aplikací pro konfiguraci a zjistit, která platforma je vhodná pro realizaci funkčního vzorku z hlediska spotřeby energie a~reakční doby.
		\subsubsection*{Klíčová slova}
		ESP8266, ESP32, měření spotřeby, měření reakčního času, JSON, HTTP, NETIO, PowerCable

	\section*{Annotation}
		The graduation thesis focuses on the comparison of the ESP8266 and ESP32 platforms. The goal is to create a driver for controlling NETIO sockets with a web application for configuration and to find out which platform is suitable for the implementation of a~functional sample in terms of energy consumption and response time.
		\subsubsection*{Keywords}
		ESP8266, ESP32, measuring usage, measuring time of reaction, JSON, HTTP, NETIO, PowerCable

	\podekovani
		Chtěl bych poděkovat panu učiteli Ing.~Břetislavu Bakalovi za odborné vedení práce a~cenné rady, které mi pomohly tuto práci zkompletovat. Rád bych také poděkoval  technickému řediteli Ing.~Břetislavu Bakalovi ml. společnosti NETIO products a.s. za cenné rady, věcné připomínky a~vstřícnost při konzultacích a~vypracování maturitní práce. V~neposlední řadě chci poděkovat Mgr. Haně Maříkové a~Mgr.~Vladimíře Špirhanzlové za~pomoc při gramatické a~stylistické kontrole.
	\tableofcontents

	\chapter{Úvod}
	%Teorie
	\chapter{Teoretický úvod do problematiky}

		\section{Netio PowerCable REST 101E}
		PoweCable REST 101E, nebo také Cobra je chytrá zásuvka s možností WiFi připojení. Díky tomu je možné zásuvku ovládat přes řadu protokolů. Zásuvka také nabízí rozhraní otevřeného API, které umožňuje integraci do systémů třetích stran. Zásuvka také dokáže měřit napětí, el. proud nebo také např. příkon. \\
		K WiFi je možné se připojit přes uživatelské rozhraní. Pro moje účely využiji odesílání JSON dat pomocí HTTP.
		\section{Platforma ESP}
			ESP jsou rodina mikročipů od společnosti \textbf{Espressif Systems}.
			\subsection{ESP8266}

				\subsubsection{Historie}
					ESP8266 je levný mikročip, který umí využívat WiFi. První chip, který se dostal na světlo světa byl v modulu \textbf{ESP-01}. Tento modul dokázal připojit se na WiFi síť a provádět jednoduché TCP/IP spojení. Získal si velkou oblibu díky nízké ceně. Jsou vhodné pro IoT jako například automatizace, zabezpečení, chytré domy atd.
				\subsubsection{Specifikace}
					Pro tuto maturitní práci bude použit modul \textbf{WT8266-S1}, který je vytvořen společností \textbf{Wireless-Tag}, protože tento modul je používán ve společnosti NETIO products a. s. Je založen na mikročipu ESP8266.\\
					ESP8266 integruje vylepšenou verzi procesoru \textbf{L106 Diamond series 32-bit} vytvořený firmou \textbf{Tensilica} s podporou frekvencí 80 \si{MHz} a 160 \si{MHz} a RTOS\footnote{Operační systém v realném čase}.
					K dispozici je 36 \si{KB} RAM a 16 \si{Mbit} Flash paměti. Intergrovaný v systému je také 10 bitový analog-digitální převodník. Modul podporuje standard {\bf IEEE802.11 b/g/n}	\footnote{Standard pro lokální bezdrátové sítě} a sadu protokolů TCP/IP. WiFi 2,4 \si{GHz} umožňuje WPA/WPA2 \footnote{Chráněný přístup k WiFi}, rychlost až 72,2 \si{Mbps} a ke komunikaci využívá anténu PCB. Zařízení má 16 GPIO pinů\footnote{Vstupně výstupní pin} (\viz{ESP8266_piny}) z toho jsou některé piny využívány na specifické činnosti. Např. pokud při bootu ESP8266 přivedeme GROUND na GPIO 0, zařízení se přepne do UART\footnote{Univerzální asynchronní přijmač, vysílač} módu a je možné nahrát do flash paměti program. Další informace jsou na obrázku \ref{ESP8266_diagram}.

					\begin{figure}[h]
						\centering
						\includegraphics[width=10cm]{ESP8266_piny.png}
						\caption{ESP8266 pinout: Literatura [1]}
						\label{ESP8266_piny}
					\end{figure}
					\begin{figure}[h]
						\centering
						\includegraphics[width=10cm]{ESP8266_diagram.png}
						\caption{ESP8266 blokový diagram funkcí: Literatura [2]}
						\label{ESP8266_diagram}
					\end{figure}

			\subsection{ESP32}
				ESP32 je mladší model z řady ESP, který byl vydán v roce 2016. Tento mikročip má integrované WiFi i Bluetooth. Série ESP32 obsahuje mikroprocesor Tensilica Xtensa LX6 v dvou jádrové či jednojádrové verzi, který může operovat na 160 \si{MHz} nebo 240 \si{MHz}. SRAM je zde byla zvetšena na 520 \si{kB}. WiFi může dosáhnout rychlosti až 150 \si{Mbps} společně se zabezpečením WPA/WPA2. Kromě IPv4 je možnost použít i IPv6. ESP32 integruje 12 bitový analog-digitální převodník. Oproti ESP8266 je do tohoto modelu zabudován senzor teploty. ESP32 disponuje 36 GPIO piny. Další funkce jsou na obrázku \ref{ESP32_diagram_pinout}. \\

					\begin{figure}[h!]
					  \centering
					  \begin{subfigure}[b]{0.4\linewidth}
					    \includegraphics[width=\linewidth]{ESP32_piny.png}
					    \caption{pinout}
					  \end{subfigure}
					  \begin{subfigure}[b]{0.4\linewidth}
					    \includegraphics[width=\linewidth]{ESP32_diagram.png}
					    \caption{blokový diagram: Literatura [3]}
					  \end{subfigure}
					  \caption{ESP32 pinout a blokový diagram}
					  \label{ESP32_diagram_pinout}
					\end{figure}

				\subsection{Deep sleep} \label{subsub:Deep sleep}
					Platforma ESP lze také uvést do úsporného režimu. Jednoduše to znamená vypnutí nedůležitých částí pro ustálený stav. ESP8266 celkově nabízí 3 druhy úsporných režimů. Modem, light a deep. Pro moji maturitní práci použiji deep sleep, který je nejúspornějších. Kromě {\bf RTC}\footnote{Hodiny realného času} se vše vypne, včetně CPU či WiFi. V tomto režimu je také možné uchovat data v tzv. RTC paměti, která jsou dostupná i po obnovení. Průměrná spotřeba je 20 \si{\micro A} a je vhodný pro jakýkoliv projekt na akumulátor či baterii. \\
					Probuzení ESP8266 probíhá, buď po nastaveném čase, kde je nutné připojit pin {\bf GPIO 16} tzv. wake-up pin na {\bf RESET} (\viz{ESP8266_timed_pressed_deepsleep}a), ten po uplynutí nastaveného času se na krátkou dobu přepne z log. 1 na log. 0 a vyresetuje ESP8266, a nebo můžeme na {\bf RESET} přivést krátce tlačítkem logickou nulu a obnovíme ESP hardwarově (\viz{ESP8266_timed_pressed_deepsleep}b).
					\begin{figure}[h!]
					  \centering-
					  \begin{subfigure}[b]{0.4\linewidth}
					    \includegraphics[width=\linewidth]{ESP8266_timed_deepsleep.png}
					    \caption{Sotwarově po daném čase}
					  \end{subfigure}
					  \begin{subfigure}[b]{0.4\linewidth}
					    \includegraphics[width=\linewidth]{ESP8266_pressed_deepsleep.png}
					    \caption{Hardwarově}
					  \end{subfigure}
					  \caption{ESP8266 probuzení z deep sleep po daném čase: Literatura [4]}
					  \label{ESP8266_timed_pressed_deepsleep}
					\end{figure}\\
					Jelikož ESP32 je novější model, deep sleep se liší v používání. ESP32 neprobouzíme přes RESET pin protože se ani na mikročipu ani nenachází. Pokud chceme ESP probudit po určitém čase, nastavíme to pouze v programu a nemusíme žádné piny propojovat. V situaci hardwarového probuzení je nutné definovat pin probuzení a při jaké logické hodnotě se má probudit. Tento pin musí být tzv. RTC\_GPIO\footnote{Pin, který je v činnosti i během spánku ESP}.
			\section{LDLN030}
			LDLN030 je napěťový regulátor, který pracuje se vstupním napětím od 1,6 V do 5,5 V. Výstupní napětí dokáže stabilizovat na 3,3 V a pokud napětí klesne pod 3 V, odpojí zařízení napojené na výstupu. Tato součástka má časté použití v mobilních telefonech, HDD a SSD discích a ostatních rařízeních napájené baterií. \\
			Regulátor má celkem 5 pinů, a to vstup, výstup, GROUND, enable pin a Power Good. Enable pin rozhoduje zda součástka funguje či nikoliv, pokud je přivedena na zem, výstup je 0 V. Power Good je pin, který může informovat například základní desku během POST\footnote{Power On Self Test}, že napájení je v pořádku a může pokračovat v činnosti.
			%https://cs.wikipedia.org/wiki/Power_Good
			\begin{figure}[h]
				\centering
				\includegraphics[width=9cm]{ldln030_datasheet.png}
				\caption{LDLN030 Pinout: Literatura [5]}
				\label{ldln030_datasheet}
			\end{figure}
			\section{Použité protokoly a jiné}
			\subsection{HTTP}
			HTTP je protokol pracující na aplikační vrstvě ISO/OSI modelu. Primárně je určen pro pro komunikaci s WWW servery a přenos hypertextových dokumentů ve formátu např. HTML. Typicky používá port TCP/80. Protokol funguje způsobem request/response\footnote{dotaz-odpověď}, kdy klient pošle request na požadovaný dokument. Server na to odpoví pomocí několika dat ohledně dotazu a poté požadovaným dokumentem.
			%https://en.wikipedia.org/wiki/Hypertext_Transfer_Protocol
			\subsection{JSON}
			JavaScript Object Notation je způsob zápisu dat nezávislých na platformě. Data mohou být organizována v polích nebo v objektech s jakýmkoliv datovým typem. JSON může složit k přenosu dat v libovolném programovacím či skriptovacím jazyku. Nevýhodou je, že nelze definovat znakovou sadu přenášeného obsahu, ale to platí jen pro některé implementace.
			%https://cs.wikipedia.org/wiki/JavaScript_Object_Notation
			\subsection{DHCP}
			Dynamic Host Configuration Protocol je protokol z rodiny TCP/IP. Je určen pro automatickou konfiguraci zařízením připojených do dané sítě. DHCP server přiděluje zařízením IP adresu, masku sítě, defaultní bránu, adresu DNS serveru a čas, kdy jsou tyto údaje platné. Tato platnost se prodlužuje pokud je na zařízení zapnut DHCP klient.\\
			Klienti zažádají server o IP adresu tak, že klient po připojení do sítě vyšle broadcastem {\bf DHCPDISCOVER} paket, na který odpoví DHCP server {\bf DHCPOFFER} s nabídkou IP adresy. Klient si z nabídek vybere jednu IP adresu a o tu požádá paketem {\bf DHCPREQUEST}. Server mu vzápětí potvrdí odpovědí {\bf DHCPACK}, když klient obdrží tuto zprávu, může IP adresu používat.
			%mirova otazka


	\chapter{Metodika postupu}
		\section{Tvorba měřecího vzorku}
			Pro měření jednotlivých scénářů jsem si vytvořil vzorek, který byl určen pouze na měření. Cílem bylo vytvořit vzorek jednoduše modifikovatelný, proto jsem zvolil pájivé pole na základní část.\\
			Na pájivém poli jsem napájel testovanou ESP platformu společně s přívodem, který byl 5 V. ESP má operační napětí 3,3 V, proto bylo nutné přidat stabilizator LDLN030, který napětí stabilizoval na operačním napětí. Stabilizátor byl zapojen dle doporučeného diagramu v datasheetu.
			\begin{figure}[h!]
				\centering
				\begin{subfigure}[b]{0.4\linewidth}
					\includegraphics[width=\linewidth]{zapojeni_esp32_vzorek.jpeg}
					\caption{Přední část}
				\end{subfigure}
				\begin{subfigure}[b]{0.4\linewidth}
					\includegraphics[width=\linewidth]{zapojeni_esp32_vzorek_zadek.jpeg}
					\caption{Zadní část}
				\end{subfigure}
				\caption{Vzorek pro měření na pajivém poli}
				\label{zapojeni_esp32_vzorek}
			\end{figure}\\
			Dále k potřebným GPIO pinům jsem napájel vodiče pro pozdější manipulaci. Pro další specifická zapojení, jako např. tlačítko, jsem použil nepájivé pole, které jsem vždy připojil specificky pro dané měření.

		\section{Měření připojení k WiFi a odeslání HTTP requestu} \label{metodika:wifi}
				Cílem měření je zjistění rychlostí a spotřeby připojení různými způsoby k přístupovému bodu a~následné porovnání případů. Do měření je také započítáné odesílání HTTP requestu zásuvce. Jsou vytvořeny celkem 4 případy akcí:
				\begin{itemize}
					\item {\bf statická IP adresa} - 	Zařízení dostane IP adresu, masku a bránu staticky staticky nakonfigurovanou. Na access pointu bude vyplé DHCP a komunikace nebude zabezpečena.
					\item {\bf dynamická IP adresa} - Bude využit DHCP protokol, kde zařízení požádá DHCP server o IP adresu, kterou mu access point přidělí společně s bránou, maskou a~s~časem, kdy tato adresa platí. Při měření nebude přístupový bod zabezpečen.
					\item {\bf Zabezpečené připojení} - Připojení na access point bude šifrované pomocí WPA2~-~PSK. DHCP server bude zapnut. Toto je simulace klasického uživatelského používání.
					\item {\bf Odeslání HTTP requestu} - Zařízení je připojené k WiFi. Odešle zprávu zásuvce a pokud dostane zpětnou vazbu od zásuvky, ukončí se činnost.
				\end{itemize}
				Ve všech scénářích bude měřeno zařízením {\bf ANALOG DISCOVERY 2} od Digilent. Frekvence procesoru je 160 \si{MHz} a AP je vzdálen 3,5 \si{m} od zásuvky a funkčního vzorku ovladače.
				\subsection{Spotřeba} \label{metodika:wifi spotreba}
					Měřen bude úbytek napětí na bočníku\footnote{nízkoohmový rezistor} o velikosti určené u měření. V programu k měřícímu zařízení je možné stejně jako na osciloskopu vytvořit graf naměřených hodnot v dané vzorkovací frekvenci. Hodnoty napětí bude převedeno na hodnoty el. proudu pomocí Ohmova zákona $I = \frac{U}{R}$. Amplitudy el. proudu budou sečteny a vyděleny počtem vzorků {\bf[DOPLNIT VZOREC]}. Tato operace nám poskytne zjištění průměrného el. proudu ve scénáři. Spotřebu již vypočítáme dle:
					$$E = U \times \overline{I} \times \frac{t}{3600}$$
					Čas {\bf t} v sekundách je možné zjistit v následující kapitole.
				\subsection{Reakční čas} \label{metodika:wifi reakce}
					Z naměřeného případu bude vyjmuta klidová část a rozdíl mezi posledním a prvním vzorkem je poté reakční čas. V případě HTTP requestu nebude počítána část konfigurace WiFi. Společně s měřením byl na {\bf serial monitor} odesílány zprávy, kde se program nachází. Poté se vyjmula část, která dle serial monitoru nepatří do scénáře.

			\section{Ustálené stavy}
				Pro ustálené stavy byly navrhnuty jednoduché programy pro co nejlepší porovnání. Ustálený stav je stav, kdy zařízení čeká na uživatelský podnět jako např. stisknutí tlačítka. Ideální ustálený stav musí splňovat:
				\begin{enumerate}
					\item nejkratší možnou reační dobu
					\item nejnižší možnou spotřebu pro co nejdelší výdrž baterií
				\end{enumerate}
				Reakční doba musí být v rámci uživatelské přívětivosti tzn. probuzení z tohoto stavu nesmí trvat více jak 1,5 \si{s}. Zároveň se zařízení nesmí vypnout zdůvodu vybité baterie za krátkou dobu. Porovnání proběhne mezi třemi scénáři, které mají jiné vlastnosti. Vybrány byly následující:
				\begin{itemize}
					\item {\bf Kontinuální režim} - V tomto stanu není žádná funkce ESP vypnuta.
					\item {\bf Deep sleep} - ESP vypne vše až na RTC (kapitola viz. \ref{subsub:Deep sleep} na straně \pageref{subsub:Deep sleep}).
					\item {\bf Enable button} - ESP je vypnuto pomocí přivedené GROUND na pin ENABLE.
				\end{itemize}


				\subsection{Spotřeba} \label{metodika:Ustálené stavy spotřeba}
					Cílem měření je nalézt stav s nejmenší spotřebou energie, jelikož finální prvek bude napájen z baterií je důležité co nejdelší výdrž. Programy, které byly napsány pro měření:\\
					\begin{itemize}
						\item {\bf Kontinuální režim} - Zařízení má nastavený určitý GPIO na tlačítko a monitoruje zda bylo zmáčknuto. ESP také má otevřenou sériovou komunikaci.
						\item {\bf Deep sleep} - Program zajišťuje probuzení ESP každých 5 \si{s} a odesílání zprávy na serial monitoring.
						\item {\bf Enable button} - Pro měření této situace není třeba specifický program
					\end{itemize}
					Pro měření bude použit {\bf ANALOG DISCOVERY} s jeho funkcí osciloskopu. Na napájení bude přidán bočník o velikosti 0,7 \si{\ohm}, 10,5 \si{\ohm} nebo 4,7 \si{\ohm} \footnote{Je nutné zvolit takový odpor, aby byl úbytek napětí měřitelný.}. Pomocí Ohmova zákona $ I = \frac{U}{R} $ je možné zjistit el. proud.
					Příkon, který bude porovnán vypočítáme dle vztahu:
					$$P = U \times I$$
				\subsection{Probuzení z ustálených časů} \label{metodika:Ustálené stavy reakce}
					Ovladač, který je již připojen k WiFi, po stisknutí tlačítka odešle HTTP request zásuvce, která zareaguje přepnutím stavu a odesláním zprávy zpět ovladači zda se povedlo či nikoliv.\\
					Reakční doba byla změřena pomocí kamery. K tlačítku jsem připojil LED, místnost jsem izoloval od světla a zmáčknutí tlačítka a reakci zásuvky jsem natočil ve zpomaleném režimu s {\bf 240 snímky za sekundu}. Pro upravení videa jsem použil trial verzi programu {\bf Sony Vegas} (\viz{sonyvegaspostup}). Našel jsem rozsvícení LED tlačítka a rozsvícení LED zásuvky ve videu a ustřihl jsem tento úsek od zbytku videa. Program poskytuje zobrazení počtu snímku daného úseku. Reakční čas je následovně možný zjistit pomocí:
					$$ t = \frac{\textrm{Počet snímků úseku}}{\textrm{Počet snímků za sekundu}}$$
					\begin{figure}[h]
						\centering
						\includegraphics[width=7cm]{sonyvegaspostup.png}
						\caption{Ukázka postupu pro měření reakčních časů}
						\label{sonyvegaspostup}
					\end{figure}

	% MĚŘENÍ
	\chapter{Měření spotřeby a času}
	Všechny výpočty jsou k nalezení v adresáři "Měření"  v souborech excel společně s použitými programy a naměřenými hodnotami v grafech.
		\section{ESP8266}
			\subsection{Spotřeba a reakční doba ustálených stavů}
				\subsubsection{Spotřeba}
					Spotřeba jednotlivých scénářů je měřena dle metodiky \ref{metodika:Ustálené stavy spotřeba} na straně \pageref{metodika:Ustálené stavy spotřeba}. Bočník pro měření úbytku napětí byl o velikosti 4,7 \si{\ohm}. \\

					\begin{figure}[h]
						\centering
						\includegraphics[width=13cm]{ESP8266_on_waiting.png}
						\caption{ESP8266 graf kontinuálního ustáleného stavu }
						\label{ESP8266_on_waiting}
					\end{figure}
					\begin{figure}[h]
						\centering
						\includegraphics[width=13cm]{ESP8266_deepsleep_waiting.png}
						\caption{ESP8266 graf deep sleep ustáleného stavu}
						\label{ESP8266_deepsleep_waiting}
					\end{figure}

					Situace {\bf enable button} nebyla možná změřit zařízením {\bf ANALOG DISCOVERY 2} ani na bočníku o velikosti 0,7 \si{\ohm} kvůli velice nízkému úbytku napětí, který se v datasheetu WT8266 pohybuje v jednotkách \si{\micro A}. V tomto případě pro výpočty je použita hodnota, kterou určil výrobce v datasheetu.

					\begin{table}[h]
						\centering
						\caption{ESP8266 porovnání spotřeby ustálených stavů}
						\begin{tabular}{||l|r r r||}
							\hline
							Ustálený stav & U($V$) & I($A$) & P($W$)\\
							\hline
							Kontinuální & $3,3$& $67,66 \times 10^{-3}$ & $223,3 \times 10^{-3}$\\
							Deep sleep& $3,3$ & $138,20 \times 10^{-6}$  & $460,0 \times 10^{-6}$\\
							Enable button & $3,3$ & $3,00\times 10^{-6}$ & $9,9\times 10^{-6}$\\
							\hline
						\end{tabular}
						\label{ESP8266 klidové režimy spotřeba}
					\end{table}
					Dle tabulky \ref{ESP8266 klidové režimy spotřeba} je možné vidět jednotlivé situace. Největším příkonem disponuje kontinuální ustálený stav z důvodu běžícího programu na kontrolu interakce. Nejmenší spotřeba je naopak vypnuté ESP přes enable pin. \\
				\subsubsection{Reakční čas}
			 		Reakční čas byl měřen dle metodiky \ref{metodika:Ustálené stavy reakce} na straně \pageref{metodika:Ustálené stavy reakce}.
				 	\begin{table}[]
					 \centering
					 \caption{ESP8266 porovnání reakčního času ustálených stavů}
					 \begin{tabular}{||l|r||}
						 \hline
						 Ustálený stav & t(ms)\\
						 \hline
						 Kontinuální & {\bf 196}\\
						 Deep sleep & {\bf 983}\\
						 Enable button & {\bf 3 208}\\
						 \hline
					 \end{tabular}
					 \label{ESP8266 klidové režimy čas}
				 	\end{table}
					Nejrychlejší ustálený stav je kontinuální, kvůli připravenosti na interakci \viztab{ESP8266 klidové režimy porovnani}. Nejdéle trvá vypnuté ESP, protože je nutné ESP zapnout, nastavit konfiguraci a připojit na WiFi, zatím co zapnuté ESP je již připraveno.
				\subsubsection{Porovnání}
					Pro porovnání spotřeb budeme předpokládat, že zařízení je napájeno 3$\times$ NiZn akumulátorem typu AA, které společně mají kapacitu {\bf 3750 mAh}, což je možné prevést na energii vztahem $E = U_{aku}\times \textrm{kapacita}$.
					a dále vztahem\footnote{$ t_{p}$ - provoz v ustáleném režimu, $ E_{bat}$ - energie 3$\times$ AA akumulátorů} $ t_{p} = \frac{E_{bat}}{P}$ zjistit provozní čas (\viztab{ESP8266 klidové režimy porovnani}). Zařízení není schopno dosáhnout této výdrže protože pokud napětí klesne pod 3 V, stabilizátor napájení odpojí.

					\begin{table}[h!]
						\centering
						\caption{ESP8266 porovnání ustálených stavů}
						\begin{tabular}{||l|r r r||}
							\hline
							Ustálený stav & P[$W$]& t$_{p}$[$h$] & t[$ms$]\\
							\hline
							Kontinuální & $223,3 \times 10^{-3}$& 75,6 & $196$\\
							Deep sleep & $460,0 \times 10^{-6}$& 36 993,1 & $983$\\
							Enable button & $9,9\times 10^{-6}$& 1 704 545,0 &  3 208\\
							\hline
						\end{tabular}
						\label{ESP8266 klidové režimy porovnani}
					\end{table}
					Kontinuální stav, zatímco má výdrž v rámci několika dní, reakční čas trvá necelých 200~ms. Enable button situace naprosto opačná. Reakční čas je pro uživatele dlouhý, ale výdrž v~tomto stavu bez interakce by se pohybovala v letech. Dobrý kompromis je situace deep sleep, kdy zařízení reagovalo vždy do jedné sekundy a teoretická výdrž je v jednotkách let.\\




			\subsection{Připojení k WiFi a odeslání HTTP requestu}
				Měření proběhlo dle metodiky \ref{metodika:wifi} na straně \pageref{metodika:wifi} na bočníku 0,7 \si{\ohm}.

				\begin{figure}[h!]
					\centering
					\begin{subfigure}[b]{0.4\linewidth}
						\includegraphics[width=\linewidth]{ESP8266_network_dynamic_specific.png}
						\caption{Dynamická IP}
					\end{subfigure}
					\begin{subfigure}[b]{0.4\linewidth}
						\includegraphics[width=\linewidth]{ESP8266_network_static.png}
						\caption{Statická IP}
					\end{subfigure}
					\begin{subfigure}[b]{0.4\linewidth}
						\includegraphics[width=\linewidth]{ESP8266_network_static_security.png}
						\caption{Zabezpečené, dynamická IP}
					\end{subfigure}
					\caption{ESP8266 grafy WiFi situací}
					\label{ESP8266_network}
				\end{figure}

				\begin{figure}[h!]
					\centering
					\includegraphics[width=12cm]{ESP8266_http.png}
					\caption{ESP8266 graf HTTP requestu}
					\label{ESP8266_http}
				\end{figure}


				\begin{table}[h]
					\centering
					\caption{ESP8266 Spotřeba operací}
					\begin{tabular}{||l| r r r r |r||}
						\hline
						Operace & U(V) & I[$mA$] & t[$ms$] & P[$mW$] & \textbf{E}[$\mu Wh$]\\
						\hline
						\hline
					Dynamické připojení & 3,3 & 89,5 & 5 300 & 295 & \textbf{435}\\
					Statické připojení & 3,3 & 83,0 & 3 143 & 274 & \textbf{239}\\
					Zabezpečené připojení & 3,3 & 90,9 & 4 606 & 300 & \textbf{383}\\
					HTTP komunikace & 3,3 & 76,7 & 483 & 253 & \textbf{34}\\
					\hline
					\end{tabular}
					\label{Spotreba_operaci}
				\end{table}
					Z tabulky \ref{Spotreba_operaci} je možné vidět, že {\bf příkon P} je téměř totožný v každé operaci. Největší rozdíl, který ovlivňuje spotřebu, je čas, za který se úkon vykoná. Jelikož {\bf připojení pomocí statické IP adresy} trvalo nejkratší dobu, má také nejmenší spotřebu energie.
					Samotná {\bf komunikace HTTP} má nejnižší spotřebu ze všech definovaných operací a jeden cyklus operace spotřebuje zanedbatelné množství energie.


		\section{ESP32}
			\subsection{Spotřeba a reakční doba ustálených stavů}
				\subsubsection{Spotřeba}
				Spotřeba jednotlivých scénářů je měřena dle metodiky \ref{metodika:Ustálené stavy spotřeba} na straně \pageref{metodika:Ustálené stavy spotřeba}. Bočník pro měření úbytku napětí byl o velikosti 10,5 \si{\ohm}.\\
				\begin{figure}[h]
					\centering
					\begin{subfigure}[b]{0.4\linewidth}
						\includegraphics[width=\linewidth]{ESP32_on_waiting.png}
						\caption{Kontinuální}
					\end{subfigure}
					\begin{subfigure}[b]{0.4\linewidth}
						\includegraphics[width=\linewidth]{ESP32_deepsleep_waiting.png}
						\caption{Deep sleep}
					\end{subfigure}
					\caption{ESP32 grafy ustálených stavů}
					\label{ESP32 waiting}
				\end{figure}
				Ustálený stav enable pin nebyl naměřen zařízením Analog Discovery 2 z důvodu velmi nízkému úbytku napětí na bočníku. K porovnání bude použita hodnota z datasheetu.
				\begin{table}[h]
					\centering
					\caption{ESP32 porovnání spotřeby ustálených stavů}
					\begin{tabular}{||l|r r r||}
						\hline
						Ustálený stav & U[$V$] & I[$A$] & P[$W$]\\
						\hline
						Kontinuální & $3,3$& $57,39 \times 10^{-3}$ & $189,4 \times 10^{-3}$\\
						Deep sleep& $3,3$ & $253,39 \times 10^{-6}$  & $838,0 \times 10^{-6}$\\
						Enable button & $3,3$ & $1,00\times 10^{-6}$ & $3,3\times 10^{-6}$\\
						\hline
					\end{tabular}
					\label{ESP32 klidové režimy spotřeba}
				\end{table}



				\subsubsection{Reakční čas}
					Získání reakčních časů jednotlivých možností zapojení ESP32 proběhlo způsobem dle \ref{metodika:Ustálené stavy spotřeba} na straně \pageref{metodika:Ustálené stavy spotřeba}. Tabulka \ref{ESP32 klidové režimy čas} ukazuje, že nejdéle trval ustálený režim v podobě vypnutého ESP32 přes enable pin a nejkratší dobu trval kontinuální ustálený stav. Kompromis mezi těmito dvěma stavy byl deep sleep ustálený stav, který se pohyboval okolo sekundy.

					\begin{table}[h]
						\centering
						\caption{ESP32 porovnání reakčního času jednotlivých situací}
						\begin{tabular}{||l|r||}
							\hline
							Ustálený stav & t[ms] \\
							\hline
							Kontinuální & {\bf 188} \\
							Enable button & {\bf 2 333} \\
							Deep sleep & {\bf 1 071} \\
							\hline
						\end{tabular}
						\label{ESP32 klidové režimy čas}
					\end{table}

				\subsection{Připojení k WiFi a odeslání HTTP requestu}
				Měření proběhlo podle metodiky \ref{metodika:wifi} na straně \pageref{metodika:wifi}.
				\begin{figure}[h!]
					\centering
					\begin{subfigure}[b]{0.4\linewidth}
						\includegraphics[width=\linewidth]{ESP32_network_dynamic_specific.png}
						\caption{Dynamická IP}
					\end{subfigure}
					\begin{subfigure}[b]{0.4\linewidth}
						\includegraphics[width=\linewidth]{ESP32_network_static.png}
						\caption{Statická IP}
					\end{subfigure}
					\begin{subfigure}[b]{0.4\linewidth}
						\includegraphics[width=\linewidth]{ESP32_network_static_security.png}
						\caption{Zabezpečené, dynamická IP}
					\end{subfigure}
					\caption{ESP32 grafy WiFi situací}
					\label{ESP32_network}
				\end{figure}
				\begin{figure}[h!]
					\centering
					\includegraphics[width=12cm]{ESP32_http.png}
					\caption{ESP832 graf HTTP requestu}
					\label{ESP32_http}
				\end{figure}

				\begin{table}[h]
					\centering
					\caption{ESP8266 Spotřeba operací}
					\begin{tabular}{||l| r r r r |r||}
						\hline
						Operace & U[V] & I[$mA$] & t[$ms$] & P[$mW$] & \textbf{E}[$\mu Wh$]\\
						\hline
						\hline
					Dynamické připojení & 3,3 & 72,0 & 1 406 & 238 & \textbf{92}\\
					Statické připojení & 3,3 & 69,2 & 1 400 & 228 & \textbf{88}\\
					Zabezpečené připojení & 3,3 & 74,7 & 1 081 & 246 & \textbf{74}\\
					HTTP komunikace & 3,3 & 62,6 & 281 & 207 & \textbf{16}\\
					\hline
					\end{tabular}
					\label{Spotreba_operaci}
				\end{table}


		\section{Porovnání získaných výsledků}
			\subsection{Ustálené stavy}
			Z tabulky \ref{Porovnání klidové režimy čas} je možné vidět jak si platformy vedly proti sobě. \\
			Pokud ovladač neustále běží s připojenou WiFi čas reakce je velmi rychlý a v obou případech uživatel nepostřehne prodlevu mezi aktivací ovladače a sepnutí relé v zásuvce. Časy mezi platformami se liší minimálně a rozdíl se pohybuje v rámci milisekund, přesto je ESP32 lehce rychlejší. \\
			Zapnutí vypnutého ESP, následné načtení a připojení k WiFi je nejdelší ze všech scénářů, který trvá několik sekund. Pro uživatele je velice pomalé a v realném produktu dle mého názoru zcela nepoužitelné. I přesto je ESP32 je mnohem rychlejší a to zhruba o 1 \si{s}. \\
			Deep sleep je alternativa mezi kontinuálním a vypnutým stavem. ESP totiž vypne vetšinu svých částí. Díky tomu je schopno relativně rychle zareagovat na budící signál. Reakce obou ovladačů je cirka 1 \si{s}. Zde vede ESP8266, které je zhruba o 100 \si{ms} rychlejší. Je to zdůvodu komplikovanější struktuře ESP32.

			\begin{table}[h]
				\centering
				\caption{Porovnání reakčních časů ústálených stavů platforem}
				\begin{tabular}{||l|r r||}
					\hline
					Ustálený stav & ESP8266 & ESP32 \\
					& [ms] & [ms] \\
					\hline
					Kontinuální & {\bf 196} & {\bf 188}\\
					Enable & {\bf 3 208} & {\bf 2 333}\\
					Deep sleep & {\bf 983} & {\bf 1 071}\\
					\hline
				\end{tabular}
				\label{Porovnání klidové režimy čas}
			\end{table}



% KONEC MĚŘENÍ


	\chapter{Vytvoření funkčních vzorků}
		\section{Tvorba uživatelského rozhraní}
		 	Veškerý kód byl napsán v textovém editoru visual studio code s pluginem Platform.io, které usnadňuje některé metody používané v Arduino IDE.\\
		 	Cílem webové stránky je umožnění připojení ovladače k uživatelské WiFi a následně vybrat pomocí IP adresy zařízení NETIO, které chceme ovládat. Stránka také musí umožňovat změnu reakce zásuvky na vyvolanou akci ovladačem.
			\section{Nastavení webserveru}
		 		Pro možnost nastavení ESP jsem vybral metodu konfigurace přes webserver, kdy uživatel se připojí na IP adresu ESP pomocí svého internetového prohlížeče. ESP webserver se zapne pouze pokud uživatel vyžádá interakcí nebo ESP není připojeno k WiFi. Pokud ESP není připojeno k WiFi, uživatel nemůže přistoupit k webové stránce, proto v tomto případě zařízení přejde do access point režimu, na který se uživatel připojí.
			\section{Připojení k WiFi}
				Jakmile uživatel přistoupí na webovou stránku, javascript požádá webserver o scannedWiFi.json. ESP poté naskenuje sítě, které se nachazí v okolí. Tyto sítě se uloží do pole tak, v jakém pořadí byly nalezeny. Pro zobrazení naskenovaných sítí na webových stránkách se vytvoří JSON string, který obsahuje počet naskenovaných sítí, pole s SSID\footnote{Identifikátor bezdrátové sítě WiFi}, pole s RSSI\footnote{Ukazatel síly signálu}, a pole, kde jsem označil zda se síť zabezpečena či nikoliv. RSSI jsem před "zabalením" do JSON pole přepočítal na sílu signálu v procentech. Po dokončení sestavení JSONu, odpoví na HTTP žádost javascriptu. Pomocí javascriptu a parametrů uložených v JSON zprávě se sestaví formuláře s jedním submit tlačítkem, které má na pozici value nastavené SSID.\\
				Pokud uživatel klikne na tlačítko s názvem dané WiFi sítě, je přesunut do adresa/wifi, kam uživatel zadá heslo. Heslo se odešle webserveru metodou POST, kde se autentizuje. Pokud síť není zabezpečena, začne probíhat připojení.
			\section{Konfigurace}
				Pro komunikaci se zásuvkou ESP potřebuje znát IP adresu zásuvky a JSON zprávu, kterou bude odesílat. Možnost konfigurace jsem uskutečnil pomocí HTML formulářů, který se metodou POST odesílá ESP. Aby ESP stále znalo uživatelskou konfiguraci i po restartu či spánku, musí si ji uložit. ESP dokáže uložit data do paměti pomocí knihovny EEPROM. Ve skutečnosti se nejedná o EEPROM paměť, která má velmi omezený počet zápisů a musí se vždy smazat celá paměť, ale ESP pouze emuluje EEPROM paměť. Ve skutečnosti tyto data ukládáme do FLASH paměti, což má řadu výhod.\\
				Pro čtení a zápis do paměti jsem vytvořil funkci.... {\bf doplnit}

	\chapter{Závěr}

	\listoftables

	\listoffigures

	\prilohy{
		\kapitola{Příloha}
	}
	% \bibliographystyle{plain}
	% \bibliography{reference}

	\literatura{
		\urll{WT8266}{WT8266}{Wireless-Tag Technology Co.,Limited}{Irvine (C)}{Wireless-Tag}{2021}{\today}{https://www.tme.com/Document/12266bdd8a30eeb152305461b089a151/WT8266-S1.pdf}

		\urll{ESP8266}{ESP8266}{Espressif Systems}{Shangai, Čína}{Espressif Systems}{2020}{\today}{https://www.espressif.com/sites/default/files/documentation/0a-esp8266ex\_datasheet\_en.pdf}

		\urll{ESP32}{ESP32}{Espressif Systems}{Shangai, Čína}{Espressif Systems}{2020}{\today}{https://www.espressif.com/sites/default/files/documentation/esp32\_datasheet\_en.pdf}

		\urll{ESP8266 Deep sleep}{ESP8266 deep sleep}{Random Nerd Tutorials}{Porto, Portugalsko}{RandomNerdTutorials.com}{2019}{\today}{https://randomnerdtutorials.com/esp8266-deep-sleep-with-arduino-ide/}

		\wiki{DHCP protokol wiki}{DHCP protokol}{2021}{\today}{https://en.wikipedia.org/wiki\_Dynamic\_Host\_Configuration\_Protocol}
	}

\end{document}
