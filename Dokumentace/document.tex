\documentclass[a4paper, 12pt]{report}
\usepackage{monapack}

\student{Milan Jiříček}
\trida{B4.I}
\obor{18-20-M/01 Informační technologie}
\bydliste{Čenkov u Bechyně 3, 391 65 Bechyně}
\datumNarozeni{10. 11. 2001}
\vedouci{Ing. Břetislav Bakala}
\nazevPrace{Dálkové ovládání zásuvek NETIO}
\cisloPrace{1}
\skolniRok{2020/2021}
\reditel{Ing. Jiří Uhlík}


\zacatek
	
	\titulniStrana

	\zadani{1. 11. 2020}
	{
		\item V dokumentaci popište technologie NETIO, ESP8266 a ESP32 s ohledem na požadované funkce ovladače
		\item Navrhněte vhodné uživatelsky přívětivé rozhraní pro konfiguraci ovladače a jeho připojení do WiFi sítě.
		\item Realizujte požadované funkce na ovladač:
		\begin{itemize}
			\item vstup na 2 nezávislá tlačítka pro standardní dvojitý přepínač/spínač dvou samostatných zásuvek jednoho zařízení NETIO, eliminace rychlých mačkání (nervózní uživatel)
			\item nepodařený příkaz/přepnutí vypínače (nepodaří se odeslat příkaz http) je indikován bliknutím LED a pípnutím buzzeru. Kontroluje se odpověď http serveru (např. 200, 404 návratové hodnoty)
			\item Zvolte vhodnou technologii výroby funkčních vzorků (nepájivé pole, DPS) a vyrobte jeden funkční vzorek na bázi ESP8266 a druhý na bázi ESP32 s možností měření odebíraného proudu
			\item Určete vhodnou metodu měření dynamické spotřeby energie v čase (reakční doby) pro různé režimy připojení, způsobu komunikace a platformy procesoru:
			\begin{itemize}
				\item ESP je vypnuto a vzbudí se nějakým externím obvodem přes enable/reset vstup po stisku tlačítka
				\item ESP je v deep sleep módu - low power režimu a vzbudí se tlačítkem
				\item ESP je kontinuálně zapnuto
				\item Kolik energie spotřebují uvedené režimy při napájení z baterie.
				Výsledky měření názorně porovnejte
			\end{itemize}		
		\end{itemize}
		\item Celou dokumentaci včetně výpisů zdrojového programu s komentářem veďte ve verzovacím systému Git a elektronicky doložte k tištěné podobě dokumentace.
	}
	{
		\item srovnání platforem ESP32 a ESP8266 s ohledem na požadované funkce
		\item návrh požadovaných funkcí ovladače
		\item návrh variant ovladače s platformou ESP32 a ESP8266 ve formě ověření konceptu
		\item zdrojové kódy v repozitáři GIT (GitHub nebo GitLab)
		\item schéma zapojeni všech zkoušených variant v řešení
		\item srovnání spotřeby a reakční doby funkčních vzorků
	}
	{
		\item vyrobené vzorky ovladače (ESP8266 a ESP32) jsou funkční
		\item je vytvořeno uživatelské rozhraní pro konfiguraci ovladače
		\item bylo provedeno měření spotřeby a doby odezvy pro oba typy ovladačů
	}
	{firma}
	{firmy}
	{}
	
	\anotace 
	Text

	\annotation
	Text
	
	\podekovani
	Text
	

	
	\obsah
	
	\kapitola{Úvod}
	\kapitola{Vlastní text práce}
		rozvedený do jednotlivých kapitol a subkapitol
		\podkapitola{Subkapitola}
			\podpodkapitola{Subsubkapitola}
	\kapitola{Závěr}

	\seznamTabulek
	
	\seznamObrazku
	
	\prilohy{
		\kapitola{Příloha}
	}
	
	\literatura{
		
		\kniha{nazevCitace}{Příjmení autora}{Jméno autora}{Název knihy}{Místo vydání}{Nakladatelství}{Rok}{ISBN}
		
		\kvalifikacniprace{nazevCitace}{Příjmení autora}{Jméno autora}{Název práce}{Místo}{Rok}{Druh práce}{Univerzita, Fakulta, Katedra}{Vedoucí diplomové práce jméno}
		
		\url{nazevCitace}{Název stránek}{Titulek}{Stránky}{rok}{datum}{URL odkaz}
		
	}
	
\konec