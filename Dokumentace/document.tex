\documentclass[a4paper, 12pt]{report}
\usepackage{monapack}
\usepackage{siunitx}
\usepackage{caption}
\captionsetup[table]{skip=5pt}

%Proměnné
\student{Milan Jiříček}
\trida{B4.I}
\obor{18-20-M/01 Informační technologie}
\bydliste{Čenkov u Bechyně 3, 391 65 Bechyně}
\datumNarozeni{10. 11. 2001}
\vedouci{Ing. Břetislav Bakala}
\nazevPrace{Dálkové ovládání zásuvek NETIO}
\cisloPrace{12}
\skolniRok{2020/2021}
\reditel{Ing. Jiří Uhlík}

%Zadání
\begin{document}

	\titulniStrana
	\anotace
		Maturitní práce se zaměřuje na porovnání platforem ESP8266 a ESP32. Cílem je vytvořit ovladač pro ovládání zásuvek značky NETIO s webovou aplikací pro konfiguraci a zjistit, která platforma je vhodná pro realizaci funkčního vzorku z hlediska spotřeby energie a~reakční doby.
	\annotation
		The graduation thesis focuses on the comparison of the ESP8266 and ESP32 platforms. The goal is to create a driver for controlling NETIO sockets with a web application for configuration and to find out which platform is suitable for the implementation of a~functional sample in terms of energy consumption and response time.
	\podekovani
		Chtěl bych poděkovat panu učiteli Ing.~Břetislavovi Bakalovi za odborné vedení práce a~cenné rady, které mi pomohly tuto práci zkompletovat. Rád bych také poděkoval  technickému řediteli Ing.~Břetislavovi Bakalovi ml. společnosti NETIO products a.s. za cenné rady, věcné připomínky a~vstřícnost při konzultacích a~vypracování bakalářské práce. V~neposlední řadě chci poděkovat Mgr. Haně Maříkové a~Mgr.~Vladimíře Špirhanzlové za~pomoc při gramatické a~stylistické kontrole.
	\tableofcontents

	\chapter{Úvod}
	%Teorie
	\chapter{Základní informace}

		\section{Zásuvka NETIO}

		\section{Platforma ESP}
			ESP jsou rodina mikročipů od společnosti \textbf{Espressif Systems} z Čínské Shangaje.
			\subsection{ESP8266}

				\subsubsection{Historie}
					ESP8266 je levný mikročip, který umí využívat WiFi. První chip, který se dostal na světlo světa byl v modulu \textbf{ESP-01}. Tento modul dokázal připojit se na WiFi síť a provádět jednoduché TCP/IP spojení. Získal si velkou oblibu u skupinek hackerů díky nízké ceně. Jsou vhodné pro IoT jako například automatizace, zabezpečení, chytré domy atd.
				\subsubsection{Specifikace}
					Pro tuto maturitní práci bude použit modul \textbf{WT8266-S1}, který je vytvořen společností \textbf{Wireless-Tag}. Je založen na mikročipu ESP8266.\\
					ESP8266 integruje vylepšenou verzi procesoru \textbf{L106 Diamond series 32-bit} vytvořený firmou \textbf{Tensilica}.

					% ESP8266 nabízí:
					% \begin{itemize}
					% 	\item 32 bitový mikroprocesor RISC architektury založen na Tensilica Xtensa Diamond L106
					% 	\item defaultně má frekvenci \SI{80}{MHz}, ale je možné nastavit i \SI{160}{MHz}
					% 	\item \SI{16}{Mb} FLASH paměť a \SI{36}{KB} RAM
					% 	\item IEEE 802.11 b/g/n, integrované zabezpečení WEP a WPA/WPA2
					% 	\item podporu $I^2C$ a $I^2S$
					% 	\item podporu Universal asynchronous reciever-transmitter(UART)
					% 	\item 16 GPIO pinů
					%
					% \end{itemize}
			\subsection{ESP32}
		\section{Komunikace mezi ESP a NETIO zásuvkou}

	\chapter{Tvorba webové stránky}

	% MĚŘENÍ
	\chapter{Měření spotřeby a času}
		\section{ESP8266}

			\subsection{Spotřeba ustálených stavů}
				Při měření spotřeby ustálených stavů bylo použito napájení z USB. Měřeno bylo zařízením \textbf{Analog Discovery 2} od společnosti \textbf{DIGILENT}. Tímto zařízením je měřeno napětí na rezistoru a dle Ohmova zákona: $I =\frac{U}{R}$ vypočítán eletrický proud. Napětí je 3,3 V.

				\subsubsection{ESP běží kontinuálně}
					Schéma zapojení \viz{ESP8266_on_schema} \\
					\obrazek{ESP8266_on_schema}{ESP8266 schéma zapojení kontinuálního ustáleného stavu}{10cm}{ESP8266_on_schema.png}
					Ústálený stav byl měřen za podmínek:
					\begin{itemize}
						\item Měřící rezistor má odpor 0,7 \si{\ohm}
						\item ESP8266 čeká na zmáčknutí tlačítka na pinu GPIO5
						\item ESP je neustále zapnuté, probíhá loop funkce pro kontrolu zmáčknutí
						\item Je připojeno k WiFi, je zaplý access point ESP, běží webserver
					\end{itemize}
		 			Při klidovém stavu byl naměřen eletrický proud průměrně 96,81 \si{mA} \viz{ESP8266_on_waiting}. Měření probíhalo 50 \si{s}. Pro jednotné porovnání je třeba vypočítat příkon:
						$$P = 96,81 \times 10^{-3}\si{A} \times 3,3 \si{ V}$$
		 			Dle rovnice se příkon rovná $ 319,5 \times 10^{-3}$ \si{\watt}\\
					\obrazek{ESP8266_on_waiting}{ESP8266 měření klidového stavu kontinualního režimu}{8cm}{ESP8266_on_waiting.png}

				\subsubsection{ESP vypnuté přes ENABLE pin}
					Schéma zapojení \viz{ESP8266_enable_schema}\\
					\obrazek{ESP8266_enable_schema}{ESP8266 schéma zapojení vypnutého ESP přes ENABLE pin}{10cm}{ESP8266_enable_schema.png}
					Měření proběhlo za podmínek:
					\begin{itemize}
						\item Měřící rezistor má odpor 10 \si{\ohm}
						\item pin enable byl připojen manuálně
					\end{itemize}
					Po připojení ESP8266 proud nevzrostl a drží se stále na 240 \si{\micro A}, což neodpovídá teoretickým hodnotám, které by se měly pohybovat okolo 3 \si{\micro A}
					\viz{1_enable}.\\
					\obrazek{1_enable}{Měření klidového režimu enable případu}{8cm}{1_enable.png}
					Pro výpočet bude jako průměrný odebraný proud použita hodnota uvedena v datasheetu což je 3 \si{\micro A}. Víme, že napětí je 3,3 \si{V} takže jsme schopni spočítat eletrický příkon:
				 		$$P = 3\times 10^{-6} \si{A}\times 3,3 \si{V}$$
					Výsledek je $9,9 \times 10^{-6}$ \si{\watt}.

				\subsubsection{Deep sleep režim}
					Schéma zapojení \viz{ESP8266_deepsleep_schema}\\
					\obrazek{ESP8266_deepsleep_schema}{ESP8266 schéma uvedené v deep sleep stavu}{10cm}{ESP8266_deepsleep_schema.png}
					Kvůli citlivosti Analog Discovery 2 nejsme schopni změřit spotřebu deep sleep režimu, je nutné změřit microampérmetrem. Pro výpočet spotřebované energie dosadíme za průměrný elektrický proud hodnotu z datasheetu, která odpovídá 20 \si{\micro A}. Spočítáme elektrický příkon:
						$$P = 20\times 10^{-6} \si{A}\times 3,3 \si{V}$$
					Ten v této situaci odpovídá hodnotě $66 \times 10^{-6}$ \si{\watt}.

				\subsubsection{Shrnutí výsledků měření spotřeby}
					\begin{table}[]
						\centering
						\caption{ESP8266 porovnání ustálených stavů}
						\begin{tabular}{||l|r r||}
							\hline
							Ustálený stav & I($A$) & P($W$)\\
							\hline
							Kontinuální & $96,81 \times 10^{-3}$ & $319,5 \times 10^{-3}$\\
							Enable & $3,00\times 10^{-6}$ & $9,9 \times 10^{-6}$\\
							Deep sleep & $20,00\times 10^{-6}$ & $66,0 \times 10^{-6}$\\
							\hline
						\end{tabular}
						\label{ESP8266 klidové režimy}
					\end{table}

			\subsection{Reakční čas jednotlivých situací}
				Reakční doba byla změřena pomocí kamery. K tlačítku jsem připojil LED, místnost jsem izoloval od světla a zmáčknutí tlačítka a reakci zásuvky jsem natočil ve zpomaleném režimu s 240 snímky za sekundu. Dále jsem zjistil rozdíl mezi rozsvícení LED u tlačítka a LED zabudované v zásuvce, signalizující sepnutí \viz{sonyvegaspostup}.
				\obrazek{sonyvegaspostup}{Ukázka postupu pro měření reakčních časů}{10cm}{sonyvegaspostup.png}

				\begin{table}[]
					\centering
					\caption{ESP8266 porovnání reakčního času jednotlivých situací}
					\begin{tabular}{||l|r||}
						\hline
						Ustálený stav & t(ms)\\
						\hline
						Kontinuální & {\bf 196}\\
						Enable & {\bf 3 100}\\
						Deep sleep & {\bf 967}\\
						\hline
					\end{tabular}
					\label{ESP8266 klidové režimy čas}
				\end{table}




				% \tabulka{ESP8266 klidové režimy čas}{Porovnání reakčního času jednotlivých situací ESP8266}
				% {& Kontinuální & Enable & Deep Sleep\\}
				% {Reakční doba  & $\SI{196}{ms}$ & $\SI{3100}{ms}$ & $\SI{967}{ms}$\\}

				\subsubsection{Porovnání reakčních časů}
					Nejrychlejší reakce byla pokud ESP8266 bylo neustále zapnuto. Nejpomalejší naopak bylo pokud ESP8266 bylo nutné zapnout, je to z důvodu načtení sketche do operační paměti, načtení konfigurace WiFi a následnému připojení \viztab{ESP8266 klidové režimy čas}.

			\subsection{Rychlost WiFi připojení}
				Cílem měření je zjistění rychlostí připojení různými způsoby k přístupovému body, spotřeby a následné porovnání případů. Všechna měření byla provedena za podmínek:

				\begin{itemize}
					\item Zařízení bylo napájeno z USB
					\item Měřeno bylo pomocí úbytku napětí na rezistoru o velokosti 0,7 \si{\ohm}
					\item Přístupový bod se nachází 3,5 \si{m} od zařízení
				\end{itemize}

				\subsubsection{Dynamické přidělení IP adresy}
					Měření proběhlo za použití DHCP protokolu, kdy zařízení požádá DHCP server o IP adresu, kterou mu Access point přidělí společně s bránou, maskou a s časem, kdy tato adresa platí. Při měření nebyl přístupový bod zabezpečen.\\
					\obrazek{ESP8266_network_dynamic}{Měření dynamického připojení k AP}{8cm}{ESP8266_network_dynamic.png}
					Měření bylo provedeno 5x. Průměrný čas se pohybuje okolo 4 652 \si{ms}. Jak je možno vidět na grafu (\viz{ESP8266_network_dynamic}), tak dvě WiFi připojení trvaly o 2 sekundy kratší dobu. ESP8266 se totiž zapíše do \textbf{DHCP client listu}, a má tak rezervovanou IP adresu, což znamená, že přiřazení proběhne rychleji.

				\subsubsection{Statické přidělení IP adresy}
					Použita byla statická adresa, která byla přidělena ESP8266 před připojením na AP. Přístupový bod nebyl zabezpečen. DHCP server byl vypnut.
					\obrazek{ESP8266_network_static}{Měření statického připojení k AP}{8cm}{ESP8266_network_static.png}
					Měření proběhlo 5x. Průměrný čas byl 3238 \si{ms}.\\
					\viz{ESP8266_network_static}

				\subsubsection{Zabezpečený AP}
						Připojení na access point je šifrované pomocí WPA2-PSK. IP adresa je na ESP nastavena staticky. DHCP server je zapnut.
						\obrazek{ESP8266_network_static_security}{Měření zabezpečeného připojení k AP}{8cm}{ESP8266_network_static_security.png}
						Průměrný čas byl 4709 \si{ms}.\\
						\viz{ESP8266_network_static_security}

				\subsubsection{Shrnutí výsledků rychlostí WiFi nastavení}
					Z výsledků měření je nejrychlejší připojení pomocí statické IP adresy, nicméně je velice náročné nastavit IP adresu, masku a bránu pro běžného uživatele. Připojení s DHCP je pomalejší průměrně o 1 \si{s} než případ se statickou IP adresou. DHCP vyniká jednoduchostí použití pro běžného uživatele. K zabezpečené WiFI trvá stejně dlouho jako s DHCP.\\ \viztab{WiFi porovnání v sekundách}

					\begin{table}[]
						\centering
						\caption{ESP8266 Porovnání reakční doby připojení k WiFi}
						\begin{tabular}{||c|c c c||}
							\hline
							Pořadí & Dynamické & Statické & Zapezpečené\\
							& (ms) & (ms) & (ms)\\
							\hline

							1. & 5 300 & 3 143 & 4 606\\
							2. & 5 300 & 3 143 & 4 606\\
							3. & 3 619 & 3 143 & 4 606\\
							4. & 5 300 & 3 143 & 4 606\\
							5. & 3 627 & 3 143 & 4 606\\
							\hline
							{\bf Průměr} & {\bf 4 629} & {\bf 3 143} & {\bf 4 606}\\
							\hline

						\end{tabular}
						\label{WiFi porovnání v sekundách}
					\end{table}


			\subsection{Rychlost odeslání HTTP requestu}
				Cílem měření je zjistit čas odesílání HTTP requestu a následné odpovězení zásuvky NETIO. Pokus byl proveden za podmínek:
				\begin{itemize}
					\item Napájeno z USB
					\item Měřeno pomocí úbytku napětí na rezistoru o velitosti 0,7 \si{\ohm}
					\item ESP8266 zkontroluje připojení k WiFi a pokud není navázáno, pokusí se ho navázat
					\item Načtení uložené konfigurace WiFi z flash paměti trvá 300 \si{ms}
					\item ESP ukončí reakci, pokud dostane zpětnou vazbu od zásuvky
				\end{itemize}
				\viz{ESP8266_http}
				\obrazek{ESP8266_http}{ESP8266 měření odesílání HTTP requestu včetně reakce zásuvky}{8cm}{ESP8266_http.png}
				Jelikož ESP přestane reagovat až po odpovězení zásuvky, dokážeme zjistit celkový čas včetně zapnutí, zkontrolování WiFi připojení, sestavení a odeslání HTTP requestu, reakce zásuvky a zpracování HTTP zprávy
				\viztab{HTTP odeslání}.
				\begin{table}[]
					\centering
					\caption{Čas odeslání HTTP requestu včetně reakce zásuvky}
					\begin{tabular}{||c|c||}
						\hline
						Pořadí & HTTP request\\
						& (ms)\\
						\hline
						\hline
						1. & 732\\
						2. & 645\\
						3. & 732\\
						4. & 732\\
						5. & 747\\
						\hline
						{\bf Průměr} & {\bf 718}\\
						\hline
					\end{tabular}
					\label{HTTP odesílání}
				\end{table}

			\subsection{Spotřeba jednotlivých operací}
				Spotřebu budeme měřit pro jednotlivé situace WiFi a odesílání HTTP requestu. Pro tyto situace využijeme data z měření v kapitolách 4.1.3 a 4.1.4. Měření WiFi připojení probíhalo za vzorkovací frekvence 160 Hz a HTTP komunikace za 68,275 Hz. Z veličin, které známe, lze vypočítat \textbf{spotřebovanou energii}:
				$$E = P \times t$$
				$$E = U \times I \times t$$

				\begin{table}[h]
					\centering
					\caption{ESP8266 Spotřeba operací}
					\begin{tabular}{||l| r r r r |r||}
						\hline
						Operace & U(V) & I($mA$) & t($ms$) & P($mW$) & \textbf{E}($\mu Wh$)\\
						\hline
						\hline
					Dynamické připojení & 3,3 & 89,5 & 5 300 & 295 & \textbf{435}\\
					Statické připojení & 3,3 & 83,0 & 3 143 & 274 & \textbf{239}\\
					Zabezpečené připojení & 3,3 & 90,9 & 4 606 & 300 & \textbf{383}\\
					HTTP komunikace & 3,3 & 84,8 & 732 & 280 & \textbf{57}\\
					\hline
					\end{tabular}
					\label{Spotreba_operaci}
				\end{table}

				\subsubsection{Shrnutí výsledků spotřeby operací}
					Z tabulky \ref{Spotreba_operaci} je možné vidět, že {\bf příkon P} je téměř totožný v každé operaci. Největší rozdíl, který ovlivňuje spotřebu, je čas, za který se úkon vykoná. Jelikož {\bf připojení pomocí statické IP adresy} trvalo nejkratší dobu, má také nejmenší spotřebu.\\
					Samotná {\bf komunikace HTTP} má nejnižší spotřebu ze všech definovaných operací a jeden cyklus operace spotřebuje zanedbatelné množství energie.
		\section{ESP32}





			% Spotřeba jednotlivých operací ESP8266 byla spočítána:
			% 	$E = U \times I \times t$
			% \tabulka{Spotřeba částí WiFi}{Spotřeba jednotlivých akcí}
			% {Operace & reakční doba & spotřeba\\}
			% {Dynamické připojení & \SI{4652}{ms} & \SI{385,19}{\micro Wh}\\
			% Statické připojení & \SI{3238}{ms} & \SI{295,59}{\micro Wh}\\
			% Zabezpečené připojení & \SI{4709}{ms} & \SI{393,15}{\micro Wh}\\
			% HTTP komunikace & \SI{718}{ms} & \SI{67,75}{\micro Wh}\\}
% KONEC MĚŘENÍ


	\chapter{Vytvoření funkčních vzorků}
	\chapter{Závěr}

	\listoftables

	\listoffigures

	\prilohy{
		\kapitola{Příloha}
	}
	\literatura{
		\urll{ESP8266}{ESP8266}{Wireless-Tag Technology Co.,Limited}{Irvine (CA)}{Wireless-Tag}{2021}{\today}{https://www.tme.com/Document/12266bdd8a30eeb152305461b089a151/WT8266-S1.pdf}
		\wiki{DHCP protokol wiki}{DHCP protokol}{2021}{\today}{https://en.wikipedia.org/wiki\_Dynamic\_Host\_Configuration\_Protocol}
	}

\end{document}
