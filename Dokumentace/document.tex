\documentclass[a4paper, 12pt]{report}
\usepackage{monapack}
\usepackage{siunitx}

%Proměnné
\student{Milan Jiříček}
\trida{B4.I}
\obor{18-20-M/01 Informační technologie}
\bydliste{Čenkov u Bechyně 3, 391 65 Bechyně}
\datumNarozeni{10. 11. 2001}
\vedouci{Ing. Břetislav Bakala}
\nazevPrace{Dálkové ovládání zásuvek NETIO}
\cisloPrace{12}
\skolniRok{2020/2021}
\reditel{Ing. Jiří Uhlík}

%Zadání
\zacatek

	\titulniStrana
	\anotace
	Maturitní práce se zaměřuje na porovnání platforem ESP8266 a ESP32. Cílem je vytvořit ovladač pro ovládání zásuvek značky NETIO s webovou aplikací pro konfiguraci a zjistit, která platforma je vhodna pro realizaci funkčního vzorku z hlediska spotřeby energie a~reakční doby.
	\annotation
	The graduation thesis focuses on the comparison of the ESP8266 and ESP32 platforms. The goal is to create a driver for controlling NETIO sockets with a web application for configuration and to find out which platform is suitable for the implementation of a~functional sample in terms of energy consumption and response time.
	\podekovani
	Chtěl bych poděkovat panu učiteli Ing.~Břetislavovi Bakalovi za odborné vedení práce a~cenné rady, které mi pomohly tuto práci zkompletovat. Rád bych také poděkoval Ing.~Břetislavovi Bakalovi za cenné rady, věcné připomínky a~vstřícnost při konzultacích a~vypracování bakalářské práce. V~neposlední řadě chci poděkovat Mgr. Haně Maříkové a~Mgr.~Vladimíře Špirhanzlové za pomoc při gramatické a~stylistické kontrole.
	\obsah
% Teorie
	\kapitola{Teorie}
		\podkapitola{Aplikace pro WiFi Managment}
		\podkapitola{Netio zásuvka Cobra}
		\podkapitola{a tak dale}
% MĚŘENÍ

	\kapitola{Měření spotřeby a času ESP8266}
	\podkapitola{Kontinuální režim}

		\podpodkapitola{Klidový stav}
			Klidový stav byl měřen za podmínek:
			\bodseznam{
		 		\bod ESP8266 čeká na zmáčknutí tlačítka na pinu GPIO5
				\bod ESP je neustále zapnuté, probíhá loop funkce pro kontrolu zmáčknutí
				\bod Je připojeno k WiFi, je zaplý soft AP, běží webserver
		 }
		 Při klidovém stavu byl naměřen eletrický proud průměrně \SI{96.81}{mA} \viz{ESP8266_on_waiting}. Měření probíhalo \SI{50}{s}. Pro teoretické vypočtení výdrže zařízení na dvou AAA bateriích je potřebné vypočítat příkon.
		 \vzorec{
			P = \SI{0.09681}{A}*\SI{3.3}{V}
		 }
		 Dle rovnice se příkon rovná \tucne{\SI{0.3195}{W}} \\
		 Pro test byly vybrány baterie \tucne{VARTA POWER AAA} o kapacitě \SI{1100}{mAh} a s napětím \SI{1.5}{V}. Jelikož známe napětí i kapacitu je možné vypočítat enegii,
		 	\vzorec{
				E = \SI{2.2}{Ah}*\SI{3}{V}
		 	}
			která se rovná {\SI{3.3}{Wh} a dále vydělíme příkonem:
			\vzorec{
			t = \zlomek{\SI{6.6}{Wh}}{\SI{0.3195}{W}}
			}
			Teoretická výdrž zařízení v kontinuálním režimu je
			\tucne{\SI{20.66}{hodin}}.
			\obrazek{ESP8266_on_waiting}{ESP8266 měření klidového stavu kontinualního režimu}{8cm}{ESP8266_on_waiting.png}

		\podpodkapitola{WiFi připojení}

			Cílem měření je zjistění rychlostí připojení různými způsoby k přístupovému body, spotřeby a následné porovnání případů.
			\podpodpodkapitola{Dynamické přidělení IP adresy}
				Měření proběhlo za použití DHCP protokolu, kde by přístupový pod měl zvolit IP adresu pro zařízení. Bylo provedeno za podmínek:
				\bodseznam{
					\bod Napájeno z USB
					\bod Měřeno pomocí úbytku napětí na rezistoru o velikosti \SI{0,7}{\ohm}
					\bod Přístupový bod nebyl zabezpečen
					\bod Přístupový bod se nachází \SI{3,5}{m} od zařízení
				}
				\obrazek{ESP8266_network_dynamic}{Měření dynamického připojení k AP}{8cm}{ESP8266_network_dynamic.png}
				% \obrazek{ESP8266_network_dynamic_specific.png}{Měření dynamického připojení k AP podrobně}{8cm}{ESP8266_network_dynamic_specific.png}
				Měření bylo provedeno 5x. Průměrný čas se pohybuje okolo \SI{4,7}{s}. Jak je možno vidět na grafu, tak dvě WiFi připojení trvaly o 2 sekundy kratší dobu. Toto chování přisuzuji rozmanitému provozu na Přístupovém bodu, který zárověň probíhá s měřením.
				\viz{ESP8266_network_dynamic}

			\podpodpodkapitola{Statické přidělení IP adresy}
			Použita byla statická adresa, která byla přidělena ESP8266 před připojením na AP. Bylo provedeno za podmínek:
			\bodseznam{
				\bod Napájeno z USB
				\bod Měřeno pomocí úbytku napětí na rezistoru o velikosti \SI{0,7}{\ohm}
				\bod Přístupový bod nebyl zabezpečen
				\bod Přístupový bod se nachází \SI{3,5}{m} od zařízení
			}
			\obrazek{ESP8266_network_static}{Měření statického připojení k AP}{8cm}{ESP8266_network_static.png}
			Měření proběhlo 5x. Průměrný čas byl \SI{3,7}{s}, což je o průměrně o sekundu rychlejší než v případě DHCP. Hodnoty grafu jsou stabilní.\\
			\viz{ESP8266_network_static}

			\podpodpodkapitola{Zabezpečený AP}
			Připojení na access point je šifrované. Bylo provedeno za podmínek:
			\bodseznam{
				\bod Napájeno z USB
				\bod Měřeno pomocí úbytku napětí na rezistoru o velikosti \SI{0,7}{\ohm}
				\bod IP adresa je nastavena staticky
				\bod Přístupový bod se nachází \SI{3,5}{m} od zařízení
				\bod Bylo použito zabezpečení WPA2-PSK
			}
			\obrazek{ESP8266_network_static_security}{Měření zabezpečeného připojení k AP}{8cm}{ESP8266_network_static_security.png}
			Průměrný čas byl \SI{4,7}{s}.\\
			\viz{ESP8266_network_static_security}

			\podpodpodkapitola{Závěr}
			Z výsledků měření je nejrychlejší připojení pomocí statické IP adresy, nicméně je velice náročné nastavit IP adresu, masku a bránu pro běžného uživatele. Připojení s DHCP je pomalejší průměrně o \SI{1}{s} než případ se statickou IP adresou. DHCP vyniká jednoduchostí použití pro běžného uživatele. K zabezpečené WiFI trvá stejně dlouho jako s DHCP.\\ \viztab{WiFi porovnání v sekundách}

			\tabulka{WiFi porovnání v sekundách}{Porovnání reakční doby naměřené připojením k WiFi}
				{Pořadí & Dynamické & Statické & Zabezpečení\\}
				{1. & \SI{5.3385}{s} & \SI{3.589}{s} & \SI{4.733}{s}\\
				2. & \SI{5.3445}{s} & \SI{3.583}{s} & \SI{4.733}{s}\\
				3. & \SI{3.619}{s} & \SI{3.631}{s} & \SI{4.733}{s}\\
				4. & \SI{5.333}{s} & \SI{3.481}{s} & \SI{4.733}{s}\\
				5. & \SI{3.627}{s} & \SI{3.613}{s} & \SI{4.733}{s}\\
				\cara
				\tucne{Průměr} & \tucne{\SI{3.5794}{s}} & \tucne{\SI{4.6524}{s}} & \tucne{\SI{4.709}{s}}\\}
			% \tabulka{WiFi porovnání v sekundách}{Porovnání reakční doby naměřené připojením k WiFi}
			% 	{veličina & Dynamické & Statické & Zabezpečení\\}
			% 	{U & \SI{3.3}{V} & \SI{3.3}{V} & \SI{3.3}{V}\\
			% 	 I & \SI{0.097}{A} & \SI{3.3}{V} & \SI{3.3}{V}\\}



		\podpodkapitola{Odeslání HTTP requestu s připojenou WiFi}
		Cílem měření je zjistit čas odesílání HTTP requestu a následné odpovězení zásuvky NETIO. Pokus byl proveden za podmínek:
		\bodseznam{
			\bod Napájeno z USB
			\bod Měřeno pomocí úbytku napětí na rezistoru o velitosti \SI{0,7}{\ohm}
			\bod ESP8266 zkontroluje připojení k WiFi a pokud není navázáno, pokusí se ho navázat
			\bod Načtení uložené konfigurace WiFi trvá \SI{300}{ms}
			\bod ESP ukončí reakci, pokud dostane zpětnou vazbu od zásuvky
		}
		Jelikož ESP přestane reagovat až po odpovězení zásuvky, dokážeme zjistit celkový čas včetně zapnutí, zkontrolování WiFi připojení, sestavení a odeslání HTTP requestu, reakce zásuvky a zpracování HTTP zprávy. \\
		\viztab{HTTP odeslání}

		\tabulka{HTTP odeslání}{Čas odeslání HTTP requestu a reakce zásuvky}
		{Pořadí & připojené k WiFi\\}
		{1. & \SI{778,9}{ms}\\
		2. & \SI{743}{ms}\\
		3. & \SI{772.9}{ms}\\
		4. & \SI{744.5}{ms}\\
		5. & \SI{623.1}{ms}\\
		\tucne{Průměr} & \tucne{\SI{732.48}{ms}}\\
		}



		\podkapitola{Enable režim}

			\podpodkapitola{Klidový stav}
				\tucne{Podmínky}
				\bodseznam{
					\bod Napájeno z USB
					\bod Měřeno pomocí úbytku napětí na rezistoru o velikosti \SI{10}{\ohm}
					\bod pin enable byl připojen manuálně
					\bod Napětí bylo měřeno Analog Discovery 2
				}
				\tucne{Výsledek} \\
				Po připojení ESP8266 proud nevzrostl a drží se stále na \SI{240}{\micro A}, což neodpovídá teoretickým hodnotám, které by se měly pohybovat okolo \SI{3}{\micro A}.
				\obrazek{1_enable}{Měření klidového režimu enable případu}{8cm}{1_enable.png}

			\podpodkapitola{WiFi připojení}
				\tucne{Podmínky}
				\bodseznam{
					\bod Měřeno pomocí úbytku napětí na rezistoru o velikosti \SI{0,7}{\ohm}
					\bod WiFi je nastavena pevně zadaná v programu
					\bod WiFi nevyužívá žádného zabezpečení
					\bod IP adresa byla nastavena staticky
				}

			\podpodkapitola{HTTP request}
				\tucne{Podmínky}
				\bodseznam{
					\bod Měřeno pomocí úbytku napětí na rezistoru o velikosti \SI{0,7}{\ohm}
					\bod WiFi je nastavena zachována v ESP z předchozího měření
				}

			\podpodkapitola{Ohodnocení výsledků}
			Výsledky klidového režimu neodpovídají teoretické hodnotě uvedené v officiálním datasheetu. Důvodem je nízká citlivost zařízení Analog Discovery 2. Pro přesnější měření je žádoucí použít micro ampérmetr. \\
			Počáteční spuštění ESP8266 trvá déle než v ostatních případech. Hlavní důvod spočívá v rozdílném načítání než v případě deep sleep... Doplním


		\podkapitola{Deep sleep režim}
			\podpodkapitola{Klidový stav}
				\tucne{Podmínky}
					\bodseznam{
						\bod Napájení z USB
						\bod Měřeno pomocí úbytku napětí na rezistoru o velikosti \SI{10}{\ohm}
						\bod ESP8266 je probuzeno každých \SI{5}{s}
					}
			\podpodkapitola{Ohodnocení výsledků}



% KONEC MĚŘENÍ
	\kapitola{Závěr}

	\seznamTabulek

	\seznamObrazku

	\prilohy{
		\kapitola{Příloha}
	}

	\literatura{

		\kniha{nazevCitace}{Příjmení autora}{Jméno autora}{Název knihy}{Místo vydání}{Nakladatelství}{Rok}{ISBN}

		\kvalifikacniprace{nazevCitace}{Příjmení autora}{Jméno autora}{Název práce}{Místo}{Rok}{Druh práce}{Univerzita, Fakulta, Katedra}{Vedoucí diplomové práce jméno}
	}

\konec
