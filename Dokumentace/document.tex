\documentclass[a4paper, 12pt]{report}
\usepackage{monapack}
\usepackage{siunitx}

%Proměnné
\student{Milan Jiříček}
\trida{B4.I}
\obor{18-20-M/01 Informační technologie}
\bydliste{Čenkov u Bechyně 3, 391 65 Bechyně}
\datumNarozeni{10. 11. 2001}
\vedouci{Ing. Břetislav Bakala}
\nazevPrace{Dálkové ovládání zásuvek NETIO}
\cisloPrace{12}
\skolniRok{2020/2021}
\reditel{Ing. Jiří Uhlík}

%Zadání
\begin{document}

	\titulniStrana
	\anotace
		Maturitní práce se zaměřuje na porovnání platforem ESP8266 a ESP32. Cílem je vytvořit ovladač pro ovládání zásuvek značky NETIO s webovou aplikací pro konfiguraci a zjistit, která platforma je vhodná pro realizaci funkčního vzorku z hlediska spotřeby energie a~reakční doby.
	\annotation
		The graduation thesis focuses on the comparison of the ESP8266 and ESP32 platforms. The goal is to create a driver for controlling NETIO sockets with a web application for configuration and to find out which platform is suitable for the implementation of a~functional sample in terms of energy consumption and response time.
	\podekovani
		Chtěl bych poděkovat panu učiteli Ing.~Břetislavovi Bakalovi za odborné vedení práce a~cenné rady, které mi pomohly tuto práci zkompletovat. Rád bych také poděkoval  technickému řediteli Ing.~Břetislavovi Bakalovi ml. společnosti NETIO products a.s. za cenné rady, věcné připomínky a~vstřícnost při konzultacích a~vypracování bakalářské práce. V~neposlední řadě chci poděkovat Mgr. Haně Maříkové a~Mgr.~Vladimíře Špirhanzlové za~pomoc při gramatické a~stylistické kontrole.
	\tableofcontents

	\chapter{Úvod}
	%Teorie
	\chapter{Základní informace}
		\section{Zásuvka NETIO}
		\section{Platforma ESP}
			ESP jsou rodina mikročipů od společnosti \textbf{Espressif Systems} z Čínské Shangaje.
			\subsection{ESP8266}
				\subsubsection{Historie}
					ESP8266 je levný mikročip, který umí využívat WiFi. První chip, který se dostal na světlo světa byl v modulu \textbf{ESP-01}. Tento modul dokázal připojit se na WiFi síť a provádět jednoduché TCP/IP spojení. Získal si velkou oblibu u skupinek hackerů díky nízké ceně.
				\subsubsection{Specifikace}
					ESP8266 nabízí:
					\begin{itemize}
						\item 32 bitový mikroprocesor RISC architektury založen na Tensilica Xtensa Diamond L106
						\item \SI{16}{Mb} flash paměť a \SI{36}{KB} RAM
						\item IEEE 802.11 b/g/n, integrované zabezpečení WEP a WPA/WPA2
						\item podporu $I^2C$ a $I^2S$
						\item 16 GPIO pinů

					\end{itemize}
			\subsection{ESP32}
	\chapter{Tvorba webové stránky}
	% MĚŘENÍ
	\chapter{Měření spotřeby a času}
		\section{ESP8266}

			\subsection{Spotřeba ustálených stavů}
				Při měření spotřeby ustálených stavů bylo použito napájení z USB. Měřeno bylo zařízením \textbf{Analog Discovery 2} od společnosti \textbf{DIGILENT}. Tímto zařízením je měřeno napětí na rezistoru a dle Ohmova zákona: $I =\frac{U}{R}$ vypočítán eletrický proud. Napětí je \SI{3.3}{V}.

				\subsubsection{ESP běží kontinuálně}
					Schéma zapojení \viz{ESP8266_on_schema} \\
					\obrazek{ESP8266_on_schema}{ESP8266 schéma zapojení kontinuálního ustáleného stavu}{10cm}{ESP8266_on_schema.png}
					Ústálený stav byl měřen za podmínek:
					\begin{itemize}
						\item Měřící rezistor má odpor \SI{0.7}{\ohm}
						\item ESP8266 čeká na zmáčknutí tlačítka na pinu GPIO5
						\item ESP je neustále zapnuté, probíhá loop funkce pro kontrolu zmáčknutí
						\item Je připojeno k WiFi, je zaplý access point ESP, běží webserver
					\end{itemize}
		 			Při klidovém stavu byl naměřen eletrický proud průměrně \SI{96.81}{mA} \viz{ESP8266_on_waiting}. Měření probíhalo \SI{50}{s}. Pro jednotné porovnání je třeba vypočítat příkon:
						$$P = 96.81 \times 10^{-3}\si{ A} \times \SI{3.3}{ V}$$
		 			Dle rovnice se příkon rovná $ 319.5 \times 10^{-3}$ \si{\watt}\\
					\obrazek{ESP8266_on_waiting}{ESP8266 měření klidového stavu kontinualního režimu}{8cm}{ESP8266_on_waiting.png}

				\subsubsection{ESP vypnuté přes ENABLE pin}
					Schéma zapojení \viz{ESP8266_enable_schema}\\
					\obrazek{ESP8266_enable_schema}{ESP8266 schéma zapojení vypnutého ESP přes ENABLE pin}{10cm}{ESP8266_enable_schema.png}
					Měření proběhlo za podmínek:
					\begin{itemize}
						\item Měřící rezistor má odpor \SI{10}{\ohm}
						\item pin enable byl připojen manuálně
					\end{itemize}
					Po připojení ESP8266 proud nevzrostl a drží se stále na \SI{240}{\micro A}, což neodpovídá teoretickým hodnotám, které by se měly pohybovat okolo \SI{3}{\micro A}
					\viz{1_enable}.\\
					\obrazek{1_enable}{Měření klidového režimu enable případu}{8cm}{1_enable.png}
					Pro výpočet bude jako průměrný odebraný proud použita hodnota uvedena v datasheetu což je \SI{3}{\micro A}. Víme, že napětí je \SI{3.3}{V} takže jsme schopni spočítat eletrický příkon:
				 		$$P = 3\times 10^{-6} \SI{}{A}\times \SI{3.3}{V}$$
					Výsledek je $9.9 \times 10^{-6}$ \si{\watt}.

				\subsubsection{Deep sleep režim}
					Schéma zapojení \viz{ESP8266_deepsleep_schema}\\
					\obrazek{ESP8266_deepsleep_schema}{ESP8266 schéma uvedené v deep sleep stavu}{10cm}{ESP8266_deepsleep_schema.png}
					Kvůli citlivosti Analog Discovery 2 nejsme schopni změřit spotřebu deep sleep režimu, je nutné změřit microampérmetrem. Pro výpočet spotřebované energie dosadíme za průměrný elektrický proud hodnotu z datasheetu, která odpovídá \SI{20}{\micro A}. Spočítáme elektrický příkon:
						$$P = 20\times 10^{-6} \SI{}{A}\times \SI{3.3}{V}$$
					Ten v této situaci odpovídá hodnotě $66 \times 10^{-6}$ \si{\watt}.

				\subsubsection{Shrnutí výsledků měření spotřeby}
					\tabulka{ESP8266 klidové režimy}{Porovnání klidových stavů ESP8266}
					{ & Kontinuální & Enable & Deep Sleep\\}
					{
					 Eletrický proud & $96.81 \times 10^{-3}$ \si{A}& $3\times 10^{-6}$ \si{A}& $20\times 10^{-6}$ \si{A}\\
					 Spotřeba & $319.5 \times 10^{-3}$ \si{\watt} & $9.9 \times 10^{-6}$ \si{\watt} & $66 \times 10^{-6}$ \si{\watt}\\
					}

			\subsection{Reakční čas jednotlivých situací}
				Reakční doba byla změřena pomocí kamery. K tlačítku jsem připojil LED, místnost jsem izoloval od světla a zmáčknutí tlačítka a reakci zásuvky jsem natočil ve zpomaleném režimu s 240 snímky za sekundu. Dále jsem zjistil rozdíl mezi rozsvícení LED u tlačítka a LED zabudované v zásuvce, signalizující sepnutí \viz{ESP8266 klidové režimy čas}.
				\tabulka{ESP8266 klidové režimy čas}{Porovnání reakčního času jednotlivých situací ESP8266}
				{& Kontinuální & Enable & Deep Sleep\\}
				{Reakční doba  & $\SI{196}{ms}$ & $\SI{3100}{ms}$ & $\SI{967}{ms}$\\}
				\subsubsection{Porovnání reakčních časů}
					Nejrychlejší reakce byla pokud ESP8266 bylo neustále zapnuto. Nejpomalejší naopak bylo pokud ESP8266 bylo nutné zapnout, je to z důvodu načtení sketche do operační paměti, načtení konfigurace WiFi a následnému připojení.

			\subsection{rychlost WiFi připojení}
				Cílem měření je zjistění rychlostí připojení různými způsoby k přístupovému body, spotřeby a následné porovnání případů.
				\subsubsection{Dynamické přidělení IP adresy}
					Měření proběhlo za použití DHCP protokolu, kde by přístupový pod měl zvolit IP adresu pro zařízení. Bylo provedeno za podmínek:
					\begin{itemize}
						\item Napájeno z USB
						\item Měřeno pomocí úbytku napětí na rezistoru o velikosti \SI{0,7}{\ohm}
						\item Přístupový bod nebyl zabezpečen
						\item Přístupový bod se nachází \SI{3,5}{m} od zařízení
					\end{itemize}
					\obrazek{ESP8266_network_dynamic}{Měření dynamického připojení k AP}{8cm}{ESP8266_network_dynamic.png}
					Měření bylo provedeno 5x. Průměrný čas se pohybuje okolo \SI{4,7}{s}. Jak je možno vidět na grafu, tak dvě WiFi připojení trvaly o 2 sekundy kratší dobu. Toto chování přisuzuji rozmanitému provozu na Přístupovém bodu, který zárověň probíhá s měřením.
					\viz{ESP8266_network_dynamic}

				\subsubsection{Statické přidělení IP adresy}
					Použita byla statická adresa, která byla přidělena ESP8266 před připojením na AP. Bylo provedeno za podmínek:
					\begin{itemize}
						\item Napájeno z USB
						\item Měřeno pomocí úbytku napětí na rezistoru o velikosti \SI{0,7}{\ohm}
						\item Přístupový bod nebyl zabezpečen
						\item Přístupový bod se nachází \SI{3,5}{m} od zařízení
					\end{itemize}
					\obrazek{ESP8266_network_static}{Měření statického připojení k AP}{8cm}{ESP8266_network_static.png}
					Měření proběhlo 5x. Průměrný čas byl \SI{3,7}{s}.\\
					\viz{ESP8266_network_static}

					\subsubsection{Zabezpečený AP}
						Připojení na access point je šifrované. Bylo provedeno za podmínek:
						\begin{itemize}
							\item Napájeno z USB
							\item Měřeno pomocí úbytku napětí na rezistoru o velikosti \SI{0,7}{\ohm}
							\item IP adresa je nastavena staticky
							\item Přístupový bod se nachází \SI{3,5}{m} od zařízení
							\item Bylo použito zabezpečení WPA2-PSK
				\end{itemize}
				\obrazek{ESP8266_network_static_security}{Měření zabezpečeného připojení k AP}{8cm}{ESP8266_network_static_security.png}
				Průměrný čas byl \SI{4,7}{s}.\\
				\viz{ESP8266_network_static_security}

			\subsubsection{Závěr}
				Z výsledků měření je nejrychlejší připojení pomocí statické IP adresy, nicméně je velice náročné nastavit IP adresu, masku a bránu pro běžného uživatele. Připojení s DHCP je pomalejší průměrně o \SI{1}{s} než případ se statickou IP adresou. DHCP vyniká jednoduchostí použití pro běžného uživatele. K zabezpečené WiFI trvá stejně dlouho jako s DHCP.\\ \viztab{WiFi porovnání v sekundách}

				\tabulka{WiFi porovnání v sekundách}{Porovnání reakční doby naměřené připojením k WiFi}
				{Pořadí & Dynamické & Statické & Zabezpečení\\}
				{1. & \SI{5.3385}{s} & \SI{3.589}{s} & \SI{4.733}{s}\\
				2. & \SI{5.3445}{s} & \SI{3.583}{s} & \SI{4.733}{s}\\
				3. & \SI{3.619}{s} & \SI{3.631}{s} & \SI{4.733}{s}\\
				4. & \SI{5.333}{s} & \SI{3.481}{s} & \SI{4.733}{s}\\
				5. & \SI{3.627}{s} & \SI{3.613}{s} & \SI{4.733}{s}\\
				\hline
				\textbf{Průměr} & \textbf{\SI{4.6524}{s}} & \textbf{\SI{3.5794}{s}} & \textbf{\SI{4.709}{s}}\\}


			\subsection{Odeslání HTTP requestu s připojenou WiFi}
				Cílem měření je zjistit čas odesílání HTTP requestu a následné odpovězení zásuvky NETIO. Pokus byl proveden za podmínek:
				\begin{itemize}
					\item Napájeno z USB
					\item Měřeno pomocí úbytku napětí na rezistoru o velitosti \SI{0,7}{\ohm}
					\item ESP8266 zkontroluje připojení k WiFi a pokud není navázáno, pokusí se ho navázat
					\item Načtení uložené konfigurace WiFi trvá \SI{300}{ms}
					\item ESP ukončí reakci, pokud dostane zpětnou vazbu od zásuvky
				\end{itemize}
				Jelikož ESP přestane reagovat až po odpovězení zásuvky, dokážeme zjistit celkový čas včetně zapnutí, zkontrolování WiFi připojení, sestavení a odeslání HTTP requestu, reakce zásuvky a zpracování HTTP zprávy. \\
				\viztab{HTTP odeslání}

				\tabulka{HTTP odeslání}{Čas odeslání HTTP requestu a reakce zásuvky}
				{Pořadí & připojené k WiFi\\}
				{1. & \SI{778,9}{ms}\\
				2. & \SI{743}{ms}\\
				3. & \SI{772.9}{ms}\\
				4. & \SI{744.5}{ms}\\
				5. & \SI{623.1}{ms}\\
				\hline
				\textbf{Průměr} & table\textbf{\SI{732.48}{ms}}\\
				}
				\tabulka{Spotřeba částí WiFi}{Spotřeba jednotlivých akcí}
				{Operace & reakční doba & spotřeba\\}
				{Dynamické připojení & \SI{4.6524}{s} & \SI{385,19}{\micro Wh}\\
				Statické připojení & \SI{3.5794}{s} & \SI{295,59}{\micro Wh}\\
				Zabezpečené připojení & \SI{4.709}{s} & \SI{393,15}{\micro Wh}\\
				HTTP komunikace & \SI{0.73248}{s} & \SI{67,75}{\micro Wh}\\}

				Spotřeba jednotlivých operací ESP8266 byla spočítána:
					$E = U \times I \times t$


% KONEC MĚŘENÍ
	\chapter{Závěr}

	\listoftables

	\listoffigures

	\prilohy{
		\kapitola{Příloha}
	}

	\literatura{

		\kniha{WT8266-S1}{Espressif Systems}{}{WT8266-S1 WiFi Module datasheet}{Shangai, Čína}{Espressif Systems}{2015}{ISBN}

		%\kvalifikacniprace{nazevCitace}{Příjmení autora}{Jméno autora}{Název práce}{Místo}{Rok}{Druh práce}{Univerzita, Fakulta, Katedra}{Vedoucí diplomové práce jméno}
		\wiki{ESP8266}{ESP8266}{2021}{\dnes}{https://en.wikipedia.org/w/index.php?title=ESP8266}
	}

\end{document}
